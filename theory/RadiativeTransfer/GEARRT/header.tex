\documentclass[12pt, a4paper, english, singlespacing, parskip]{scrartcl}

%\documentclass[
%11pt, 				% The default document font size, options: 10pt, 11pt, 12pt
%oneside, 			% Two side (alternating margins) for binding by default, uncomment to switch to one side
%chapterinoneline,	% Have the chapter title next to the number in one single line
%english, 			% ngerman for German
%singlespacing, 	% Single line spacing, alternatives: onehalfspacing or doublespacing
%draft, 			% Uncomment to enable draft mode (no pictures, no links, overfull hboxes indicated)
%nolistspacing, 	% If the document is onehalfspacing or doublespacing, uncomment this to set spacing in lists to single
%liststotoc, 		% Uncomment to add the list of figures/tables/etc to the table of contents
%toctotoc, 			% Uncomment to add the main table of contents to the table of contents
%parskip, 			% Uncomment to add space between paragraphs
%nohyperref, 		% Uncomment to not load the hyperref package
%headsepline, 		% Uncomment to get a line under the header
%]{scrartcl or scrreprt or scrbook} % The class file specifying the document structure

\usepackage{lmodern} 		% Diese beiden packages sorgen für echte
\usepackage[T1]{fontenc}	% Umlaute.

\usepackage{amssymb, amsmath, color, graphicx, float, setspace, tipa}
\usepackage[utf8]{inputenc}
\usepackage[english]{babel}
\usepackage[pdfpagelabels,
			pdfstartview = FitH,
			bookmarksopen = true,
			bookmarksnumbered = true,
			linkcolor = black,
			plainpages = false,
			hypertexnames = false,
			citecolor = black,
			breaklinks]{hyperref}
\usepackage{url}
\usepackage{array}

\allowdisplaybreaks 		% allows page breaks in align/equation environment

\usepackage{authblk} 		% titlepage stuff
\usepackage[titletoc, title]{appendix}

\usepackage{newclude} 		% use \include*{file} instead \include{} to omit pagebreak after include
\usepackage{lscape}




%===================
% BIBLIOGRAPHY
%===================

\usepackage[]{natbib}
\bibliographystyle{apalike}

%DONT FORGET TO COMPILE THE BIBLIOGRAPHY WITH BIBTEX WHEN CHANGES ARE MADE.



%--------------------------------------------
%     OPTIONAL
%--------------------------------------------



%% Change font
%\newcommand{\changefont}[3]{
%\fontfamily{#1} \fontseries{#2} \fontshape{#3} \selectfont}
%\changefont{ppl}{m}{n} nach \begin{document} einsetzen

% Fig. instead of Figure, Tab. instead of Table
%\usepackage[footnotesize]{caption2}
%\addto\captionsenglish{\renewcommand{\figurename}{Fig.}}
%\addto\captionsngerman{\renewcommand{\figurename}{Fig.}}
%\renewcommand{\tablename}{Tab.}%

%\pagestyle{headings} % Write headings on each page

%\usepackage{chngcntr} \counterwithout{figure}{section} % Integer only figure numbers, ignoring chapter numbers






%-----------------------------------------------
% FORMAT TITLE
%-----------------------------------------------


% Set fonts of document parts
\setkomafont{title}{\rmfamily\bfseries\boldmath}
\addtokomafont{section}{\rmfamily\bfseries\boldmath}
\addtokomafont{subsection}{\rmfamily\bfseries\boldmath}
\addtokomafont{subsubsection}{\rmfamily\bfseries\boldmath}
\addtokomafont{disposition}{\rmfamily} % table of contents and stuff
\setkomafont{descriptionlabel}{\rmfamily\bfseries\boldmath}


\usepackage{dsfont} 	% for Pressure tensor P with \mathds{}




%-----------------------------------------------
% Document specific definitions
%-----------------------------------------------

\newcommand{\Aij}{$\mathcal{A}_{ij}$}	% A_ij
\newcommand{\Aijm}{\ensuremath{\mathcal{A}_{ij}}}	% A_ij math
\newcommand{\U}{\ensuremath{\mathcal{U}}}			% State vector
\newcommand{\W}{\ensuremath{\mathcal{W}}}			% State vector
\newcommand{\F}{\ensuremath{\mathcal{F}}}			% Flux tensor
\newcommand{\Ubf}{\ensuremath{\mathbf{U}}}			% State vector
\newcommand{\Fbf}{\ensuremath{\mathbf{F}}}			% Flux tensor
\newcommand{\psitilde}{\ensuremath{\tilde{\psi}}}	% psi tilde
\newcommand{\half}{1/2}                 % 1/2


\newcommand{\absorbers}{\text{HI, HeI, HeII}}
\newcommand{\Ndot}{\dot{N}}


%-----------------------------------------------
% General Lazyness
%-----------------------------------------------

\newcommand{\corresponds}{\mathrel{\widehat{=}}}       % equals with hat

\newcommand {\arctanh}{\mathrm{arctanh}}               % Atanh
\newcommand{\arccot}{\mathrm{arccot }}                 % Acotanh

\newcommand{\limz}[1]{\lim\limits_{#1 \rightarrow 0}}  % Limes of something towards zero

\newcommand{\bm}{\boldmath}                            % Bold font in math
\newcommand{\dps}{\displaystyle}

\newcommand{\e}{\mbox{e}}                              % e noncursive in math mode

\newcommand{\del}{\partial}                            % partial diff operator
\newcommand{\de}{\mathrm{d}}                           % differential d
\newcommand{\D}{\mathrm{d}}                            % differential d
\newcommand{\GRAD}{\mathrm{grad}\ }                    % gradient
\newcommand{\DIV}{\mathrm{div}\ }                      % divergence
\newcommand{\ROT}{\mathrm{rot}\ }                      % rotation

\newcommand{\CONST}{\mathrm{const.\ }}                 % constant
\newcommand{\var}{\mathrm{var}}                        % variance

\newcommand{\g}{^\circ}                                % degrees
\newcommand{\degr}{^\circ}                             % degrees

\newcommand{\msol}{M_\odot}                            % solar mass
\newcommand{\Lsol}{L_\odot}                            % solar luminosity
\newcommand{\Lsun}{L_\odot}                            % solar luminosity
\newcommand{\order}{\mathcal{O}}                       % order, e.g. O(h^2)


\newcommand{\x}{\mathbf{x}}                            % x vector
\newcommand{\xdot}{\dot{\mathbf{x}}}                   % x dot vector
\newcommand{\xddot}{\ddot{\mathbf{x}}}                 % x doubledot vector
\newcommand{\R}{\mathbf{r}}                            % r vector
\newcommand{\rdot}{\dot{\mathbf{r}}}                   % r dot vector
\newcommand{\rddot}{\ddot{\mathbf{r}}}                 % r doubledot vector
\newcommand{\vel}{\mathbf{v}}                          % v vector
\newcommand{\V}{\mathbf{v}}                            % v vector
\newcommand{\vdot}{\dot{\mathbf{v}}}                   % v dot vector
\newcommand{\vddot}{\ddot{\mathbf{v}}}                 % v doubledot vector

\newcommand{\dete}{\mathrm{d}t}                        % dt
\newcommand{\delte}{\del t}                            % partial t
\newcommand{\dex}{\mathrm{d}x}                         % dx
\newcommand{\delx}{\del x}                             % partial x
\newcommand{\der}{\mathrm{d}r}                         % dr
\newcommand{\delr}{\del r}                             % partial r


\newcommand{\deldx}{\frac{\del}{\del x}}				% shortcut partial derivative, in line
\newcommand{\ddx}{\frac{\de}{\de x}}					% shortcut total derivative, in line
\newcommand{\DELDX}[1]{\frac{\del  #1}{\del x}}			% shortcut partial derivative, on fraction
\newcommand{\DDX}[1]{\frac{\de  #1}{\de x}}				% shortcut total derivative, on fraction

\newcommand{\deldvecx}{\frac{\del}{\del \x}}	   		% shortcut partial derivative, in line
\newcommand{\ddvecx}{\frac{\de}{\de \x}}				% shortcut total derivative, in line
\newcommand{\DELDVECX}[1]{\frac{\del  #1}{\del \x}}		% shortcut partial derivative, on fraction
\newcommand{\DDVECX}[1]{\frac{\de  #1}{\de \x}}			% shortcut total derivative, on fraction

\newcommand{\deldr}{\frac{\del}{\del r}}				% shortcut partial derivative, in line
\newcommand{\ddr}{\frac{\de}{\de r}}					% shortcut total derivative, in line
\newcommand{\DELDR}[1]{\frac{\del  #1}{\del r}}			% shortcut partial derivative, on fraction
\newcommand{\DDR}[1]{\frac{\de  #1}{\de r}}				% shortcut total derivative, on fraction

\newcommand{\deldt}{\frac{\del}{\del t}}				% shortcut partial derivative, in line
\newcommand{\ddt}{\frac{\de}{\de t}}					% shortcut total derivative, in line
\newcommand{\DELDT}[1]{\frac{\del  #1}{\del t}}			% shortcut partial derivative, on fraction
\newcommand{\DDT}[1]{\frac{\de  #1}{\de t}}				% shortcut total derivative, on fraction

\newcommand{\deldxalpha}{\frac{\del}{\del x^\alpha}}
\newcommand{\DELDXALPHA}[1]{\frac{\del #1}{\del x^\alpha}}
\newcommand{\ddxalpha}{\frac{\de}{\de x^\alpha}}
\newcommand{\DDXALPHA}[1]{\frac{\de #1}{\de x^\alpha}}





